\documentclass[a4paper]{article}

% if you need to pass options to natbib, use, e.g.:
%     \PassOptionsToPackage{numbers, compress}{natbib}
% before loading neurips_2022


% ready for submission
%\usepackage{ofu_xai_2022}

\input{math_commands.tex}

% to compile a preprint version, e.g., for submission to arXiv, add add the
% [preprint] option:
%     \usepackage[preprint]{ofu_xai_2022}


% to compile a camera-ready version, add the [final] option, e.g.:
\usepackage[final, nonatbib]{ofu_xai_2022}

% to avoid loading the natbib package, add option nonatbib:
%    \usepackage[nonatbib]{ofu_xai_2022}


\usepackage[utf8]{inputenc} % allow utf-8 input
\usepackage[T1]{fontenc}    % use 8-bit T1 fonts
\usepackage{hyperref}       % hyperlinks
\usepackage{url}            % simple URL typesetting
\usepackage{booktabs}       % professional-quality tables
\usepackage{amsfonts}       % blackboard math symbols
\usepackage{nicefrac}       % compact symbols for 1/2, etc.
\usepackage{microtype}      % microtypography
\usepackage{xcolor}         % colors

\usepackage[round]{natbib}
%%%%

\title{ Your Awesome Title\\ {\large xAI-Proj-B: Bachelor Project Explainable Machine Learning }}


% The \author macro works with any number of authors. There are two commands
% used to separate the names and addresses of multiple authors: \And and \AND.
%
% Using \And between authors leaves it to LaTeX to determine where to break the
% lines. Using \AND forces a line break at that point. So, if LaTeX puts 3 of 4
% authors names on the first line, and the last on the second line, try using
% \AND instead of \And before the third author name.


\author{%
  Max Mustermann\thanks{Degree: M.Sc. AI, matriculation \#: 12345678} \\
  Otto-Friedrich University of Bamberg\\
  96049 Bamberg, Germany\\
  \texttt{max.mustermann@stud.uni-bamberg.de}\\
  % examples of more authors
   \And
   Erika Musterfrau\thanks{Degree: M.Sc. AI, matriculation \#: 12345678}\\
   Otto-Friedrich University of Bamberg\\
   96049 Bamberg, Germany\\
   \texttt{erika.musterfrau@stud.uni-bamberg.de} \\
   \And
   John Doe\thanks{Degree: M.Sc. AI, matriculation \#: 12345678}\\
   Otto-Friedrich University of Bamberg\\
   96049 Bamberg, Germany\\
   \texttt{john.doe@stud.uni-bamberg.de} \\
  % \And
  % Coauthor \\
  % Affiliation \\
  % Address \\
  % \texttt{email} \\
  % \And
  % Coauthor \\
  % Affiliation \\
  % Address \\
  % \texttt{email} \\
}


\begin{document}


\maketitle
\def\va{{\bm{a}}}

\begin{abstract}
  The abstract paragraph should be indented \nicefrac{1}{2}~inch on
  both the left- and right-hand margins. Use 10~point type, with a vertical
  spacing (leading) of 11~points.  The word \textbf{Abstract} must be centered,
  bold, and in point size 12. Two line spaces precede the abstract. The abstract
  must be limited to one paragraph.
\end{abstract}


\section{Submission of project works xAI-Proj}


Please read the instructions below carefully and follow them faithfully. Those instructions and style file is based on \verb+neurips_2022.tex+ (LPPL v1.3c) and was adapted for the purposes at the Chair of Explainable Machine Learning at the University of Bamberg. Please indicate who wrote which part of the project report. Also, include the link to your git repository.


\subsection{Style}


Papers to be submitted must be prepared according to the
instructions presented here. Papers may be in the range of {\bf 12-15} pages long,
including figures. Additional pages \emph{containing only acknowledgments and
references} are allowed. Papers that exceed the page limit will not be
accepted.


Authors are required to use the the \verb+ofu_xai_2022.sty+ \LaTeX{} style file. Tweaking the style files may be grounds for rejection.

The file \verb+ofu_xai_2022.tex+ may be used as a ``shell'' for writing your
paper. All you have to do is replace the author, title, abstract, and text of
the paper with your own.


The formatting instructions contained in these style files are summarized in
Sections \ref{gen_inst}, \ref{headings}, and \ref{others} below.


\section{General formatting instructions}
\label{gen_inst}


The text must be confined within a rectangle 5.5~inches wide and
9~inches long. The left margin is 1.5~inch.  Use 10~point
type with a vertical spacing (leading) of 11~points.  Times New Roman is the
preferred typeface throughout, and will be selected for you by default.
Paragraphs are separated by \nicefrac{1}{2}~line space (5.5 points), with no
indentation.


The paper title should be 17~point, initial caps/lower case, bold, centered
between two horizontal rules. The top rule should be 4~points thick and the
bottom rule should be 1~point thick. Allow \nicefrac{1}{4}~inch space above and
below the title to rules. All pages should start at 1~inch from the
top of the page.


For the final version, authors' names are set in boldface, and each name is
centered above the corresponding address. The lead author's name is to be listed
first (left-most), and the co-authors' names (if different address) are set to
follow. If there is only one co-author, list both author and co-author side by
side.


Please pay special attention to the instructions in Section \ref{others}
regarding figures, tables, acknowledgments, and references.


\section{Headings: first level}
\label{headings}


All headings should be lower case (except for first word and proper nouns),
flush left, and bold.


First-level headings should be in 12-point type.


\subsection{Headings: second level}


Second-level headings should be in 10-point type.


\subsubsection{Headings: third level}


Third-level headings should be in 10-point type.


\paragraph{Paragraphs}


There is also a \verb+\paragraph+ command available, which sets the heading in
bold, flush left, and inline with the text, with the heading followed by 1\,em
of space.


\section{Citations, figures, tables, references}
\label{others}


These instructions apply to everyone.


\subsection{Citations within the text}


The \verb+natbib+ package will be loaded for you by default.  Citations may be
author/year or numeric, as long as you maintain internal consistency.  As to the
format of the references themselves, any style is acceptable as long as it is
used consistently.


The documentation for \verb+natbib+ may be found at
\begin{center}
  \url{http://mirrors.ctan.org/macros/latex/contrib/natbib/natnotes.pdf}
\end{center}
Of note is the command \verb+\citet+, which produces citations appropriate for
use in inline text.  For example,
\begin{verbatim}
   \citet{He_2016_CVPR} investigated\dots
\end{verbatim}
produces
\begin{quote}
  \citet{He_2016_CVPR} investigated\dots
\end{quote}

For standard reference the command \verb+\citep+ is appropriate and produces \citep{He_2016_CVPR}. Multiple references can be cited e.g. with
\begin{verbatim}
    \citep{Bengio_chapter2007,He_2016_CVPR,Hinton06,goodfellow2016deep}
\end{verbatim}
yielding 
\begin{quote}
    \citep{Bengio_chapter2007,He_2016_CVPR,Hinton06,goodfellow2016deep}
\end{quote}

If you wish to load the \verb+natbib+ package with options, you may add the
following before loading the \verb+ofu_xao_2022+ package:
\begin{verbatim}
   \PassOptionsToPackage{options}{natbib}
\end{verbatim}


If \verb+natbib+ clashes with another package you load, you can add the optional
argument \verb+nonatbib+ when loading the style file:
\begin{verbatim}
   \usepackage[nonatbib]{ofu_xao_2022}
\end{verbatim}


\subsection{Footnotes}


Footnotes should be used sparingly.  If you do require a footnote, indicate
footnotes with a number\footnote{Sample of the first footnote.} in the
text. Place the footnotes at the bottom of the page on which they appear.
Precede the footnote with a horizontal rule of 2~inches.


Note that footnotes are properly typeset \emph{after} punctuation
marks.\footnote{As in this example.}


\subsection{Figures}


\begin{figure}
  \centering
  \fbox{\rule[-.5cm]{0cm}{4cm} \rule[-.5cm]{4cm}{0cm}}
  \caption{Sample figure caption.}
\end{figure}


All artwork must be neat, clean, and legible. Lines should be dark enough for
purposes of reproduction. The figure number and caption always appear after the
figure. Place one line space before the figure caption and one line space after
the figure. The figure caption should be lower case (except for first word and
proper nouns); figures are numbered consecutively.


You may use color figures.  However, it is best for the figure captions and the
paper body to be legible if the paper is printed in either black/white or in
color.


\subsection{Tables}


All tables must be centered, neat, clean and legible.  The table number and
title always appear before the table.  See Table~\ref{sample-table}.


Place one line space before the table title, one line space after the
table title, and one line space after the table. The table title must
be lower case (except for first word and proper nouns); tables are
numbered consecutively.


Note that publication-quality tables \emph{do not contain vertical rules.} We
strongly suggest the use of the \verb+booktabs+ package, which allows for
typesetting high-quality, professional tables:
\begin{center}
  \url{https://www.ctan.org/pkg/booktabs}
\end{center}
This package was used to typeset Table~\ref{sample-table}.


\begin{table}
  \caption{Sample table title}
  \label{sample-table}
  \centering
  \begin{tabular}{lll}
    \toprule
    \multicolumn{2}{c}{Part}                   \\
    \cmidrule(r){1-2}
    Name     & Description     & Size ($\mu$m) \\
    \midrule
    Dendrite & Input terminal  & $\sim$100     \\
    Axon     & Output terminal & $\sim$10      \\
    Soma     & Cell body       & up to $10^6$  \\
    \bottomrule
  \end{tabular}
\end{table}


\section{Final instructions}


Do not change any aspects of the formatting parameters in the style files.  In
particular, do not modify the width or length of the rectangle the text should
fit into, and do not change font sizes (except perhaps in the
\textbf{References} section; see below). Please note that pages should be
numbered.


\section{Preparing PDF files}


Please prepare submission files with paper size ``A4,'' and not, for
example, ``Letter''.


Fonts can be the main cause of problems. Your PDF file should only
contain Type 1 or Embedded TrueType fonts.



\subsection{Margins in \LaTeX{}}


Most of the margin problems come from figures positioned by hand using
\verb+\special+ or other commands. We suggest using the command
\verb+\includegraphics+ from the \verb+graphicx+ package. Always specify the
figure width as a multiple of the line width as in the example below:
\begin{verbatim}
   \usepackage[pdftex]{graphicx} ...
   \includegraphics[width=0.8\linewidth]{myfile.pdf}
\end{verbatim}
See Section 4.4 in the graphics bundle documentation
(\url{http://mirrors.ctan.org/macros/latex/required/graphics/grfguide.pdf})


A number of width problems arise when \LaTeX{} cannot properly hyphenate a
line. Please give LaTeX hyphenation hints using the \verb+\-+ command when
necessary.


% \begin{ack}
% Use unnumbered first level headings for the acknowledgments. All acknowledgments
% go at the end of the paper before the list of references. Moreover, you are required to declare
% funding (financial activities supporting the submitted work) and competing interests (related financial activities outside the submitted work).
% More information about this disclosure can be found at: \url{https://neurips.cc/Conferences/2022/PaperInformation/FundingDisclosure}.


% Do {\bf not} include this section in the anonymized submission, only in the final paper. You can use the \texttt{ack} environment provided in the style file to autmoatically hide this section in the anonymized submission.
% \end{ack}

\section{Notation}
\input{math_notation.tex}

\section*{References}

Any choice of citation style is acceptable as long as you are
consistent. It is permissible to reduce the font size to \verb+small+ (9 point)
when listing the references.
Note that the Reference section does not count towards the page limit.
\medskip


\bibliography{bibliography}
\bibliographystyle{abbrvnat}


%%%%%%%%%%%%%%%%%%%%%%%%%%%%%%%%%%%%%%%%%%%%%%%%%%%%%%%%%%%%%%%%%%%%%%%%%%%%%%%%%%%%%%%%%%%%%%%%%%%%
%% Declaration of Authorship
%%%%%%%%%%%%%%%%%%%%%%%%%%%%%%%%%%%%%%%%%%%%%%%%%%%%%%%%%%%%%%%%%%%%%%%%%%%%%%%%%%%%%%%%%%%%%%%%%%%%

\section*{Declaration of Authorship}
All final papers have to include the following ‘Declaration of Authorship’:

{\parindent 0cm
%%%%%%%%%%%%%%%%%%%%%%%%%%German%%%%%%%%%%%%%%%%%%%%%%%%%%%%%%
\section*{Declaration of Authorship}
Ich erkläre hiermit gemäß § 9 Abs. 12 APO, dass ich die vorstehende Projektarbeit selbstständig verfasst und keine anderen als die angegebenen Quellen und Hilfsmittel benutzt habe. Des Weiteren erkläre ich, dass die digitale Fassung der gedruckten Ausfertigung der Projektarbeit ausnahmslos in Inhalt und Wortlaut entspricht und zur Kenntnis genommen wurde, dass diese digitale Fassung einer durch Software unterstützten, anonymisierten Prüfung auf Plagiate unterzogen werden kann.\\
\vspace{2\baselineskip}
  
Bamberg, \today

\rule[0.5em]{14em}{0.5pt} \hspace{0.25\linewidth}\rule[0.5em]{14em}{0.5pt}
\vspace{1em}
\hspace{4em} (Place, Date) \hspace{0.51\linewidth} (Signature)

Bamberg, \today

\rule[0.5em]{14em}{0.5pt} \hspace{0.25\linewidth}\rule[0.5em]{14em}{0.5pt}
\vspace{1em}
\hspace{4em} (Place, Date) \hspace{0.51\linewidth} (Signature)

Bamberg, \today

\rule[0.5em]{14em}{0.5pt} \hspace{0.25\linewidth}\rule[0.5em]{14em}{0.5pt}
\vspace{1em}
\hspace{4em} (Place, Date) \hspace{0.51\linewidth} (Signature)
}



%%%%%%%%%%%%%%%%%%%%%%%%%%%%%%%%%%%%%%%%%%%%%%%%%%%%%%%%%%%%


%%%%%%%%%%%%%%%%%%%%%%%%%%%%%%%%%%%%%%%%%%%%%%%%%%%%%%%%%%%%


\appendix


\section{Appendix}


Optionally include extra information (complete proofs, additional experiments and plots) in the appendix.
This section will often be part of the supplemental material.


\end{document}

\section{Conclusion}\label{conclusion}

In this report, we have explored the optimization and training of various CNN architectures on subsets of the MNIST and MedMNIST datasets. To improve our results, we tested multiple optimization techniques: class balancing, data augmentation and batch normalization. Of these methods, data augmentation proved to be effective in reducing the rate of false positives, improving the rate of true positives, and overall improving the accuracy of ResNet models by, in some cases, multiple percentage points. On the other hand, tests on AlexNet showed the effectiveness of data balancing. Gathering significant performance improvements while not being impacted by data augmentation or batch normalization. This approach, however, should be used with caution when used for real-world medical applications, as none of us have the required medical expertise to judge how this kind of tempering with the data might translate to broader and real-life data. While a simple CNN worked well for the classification of hand-drawn numbers from the MNIST dataset, it struggled with the more complex data in MedMNIST\@. Of our employed model architectures, ResNet and Xception proved most effective on the MedMNIST dataset, proving the value of deeper model architectures in complex classification tasks. Some of our approaches and results, especially for ResNet and ResNet related testing, might have been impacted by inconsistencies due to the non-deterministic nature of training using cuda and the lack of preventive measures on our part to prevent or minimize it. For the future, it would be great to test even more architectures and potentially fine-tune them even further for the particular use case. It would also be interesting to see, how the observed performance differences between different sizes of one model architecture scale with a larger dataset. By tackling the classification of tissue patterns, we have not only explored some technical intricacies of deep learning, but also contributed to an area where such technologies could potentially revolutionize diagnostic methods.

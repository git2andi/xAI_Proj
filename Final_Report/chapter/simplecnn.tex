\subsubsection{SimpleCNN}\label{simplecnn}

Our SimpleCNN model contains two primary convolutional layers. The choice of this architecture was motivated by the goal of understanding the basic mechanisms of neural networks in processing and classifying image data. The model uses LeakyReLU activation to avoid the vanishing gradient problem and to ensure effective backpropagation even at small gradient values. The first convolution layer uses 32 filters with a kernel size of 3$\times$3, a stride of 1 and `same' padding, which preserves the dimension of the input images. This is followed by a max-pooling layer with a kernel size of 2, which aims to reduce the spatial dimensions while keeping the important features to optimize the model's ability to detect patterns without considerable data loss. Subsequently, a second convolution sequence increases the depth to 64 filters, improving the model's ability to extract more features. This setup is again followed by LeakyReLU activation and max-pooling, further refining the feature extraction process. The architecture concludes with a linear layer that maps the high-level features to the output classes, with a softmax function to interpret the outputs as class probabilities.

As our project progressed and our understanding deepened, we enhanced the initial SimpleCNN model by introducing an additional convolutional layer and incorporating batch normalization and dropout techniques, aiming for improved accuracy and generalization. The modified SimpleCNN architecture begins with a convolutional layer designed for single-channel (grayscale) images to match the format of the MNIST dataset. This layer, consisting of 32 filters with a kernel size of 3$\times$3 and `same' padding, is followed by batch normalization and ReLU activation, which promotes nonlinearity while maintaining normalization across the inputs of the network. A dropout rate of 0.25 after max pooling aims to mitigate overfitting by randomly omitting a portion of the features during the training process. As the model progresses, the second and third convolutional layers increase the filter count to 64 and then 128, respectively, each augmented with batch normalization, ReLU activation, max pooling, and a consistent dropout rate of 0.25. This architectural depth ensures the extraction of increasingly complex features essential for recognizing diverse patterns in handwritten digits. The concluding segment of the model comprises a fully connected layer transitioning from the convolutional output to 256 units, followed by batch normalization and a higher dropout rate of 0.5, further combating overfitting. The final linear layer maps these processed features to the ten class outputs corresponding to the MNIST dataset's digit categories.

The transition from the SimpleCNN model, which was tailored for the MNIST dataset, to the model customized for the PathMNIST challenge required some adjustments to account the different characteristics of the two datasets. The most important change was to adapt the input layer to include 3-channel RGB images for PathMNIST\@. This change is important to take advantage of the color information that is critical in medical imaging for identifying different tissue types. Despite this adaptation, the core architecture~-~consisting of convolutional layers, batch normalization, ReLU activation, and dropout layers~-~remained consistent between models.

For a detailed implementation of all versions of the model, refer to the Appendix Section~\ref{codeSnippets}.\@
\section{Introduction}\label{intro}

The core of this project was to understand the key principles of deep learning and to apply them in a practical environment. Through the methodical training and evaluation of various techniques, we progressively deepened our understanding of this complex field. A crucial first step in developing an effective classification model was to thoroughly investigate and understand the dataset at hand. Therefore, our investigation began with an introduction to the well-known datasets MNIST \citep{deng2012mnist} and MedMNIST \citep{medmnistv1}, which served as the building blocks for our study.

The project was structured to take account of the different characteristics of the individual datasets. We started with MNIST, which was chosen due to its wide distribution and the numerous tutorials available, which facilitated our entry into the world of deep learning. With this dataset, we took on the challenge of developing a rudimentary Convolutional Neural Network (SimpleCNN) that was intentionally designed with a limited number of layers. The initial aim of this challenge was not to achieve peak performance, but rather to gain practical experience and understand the basics of the architecture of neural networks and their ability to distinguish between different digits.

As our knowledge increased, we shifted our focus to the more challenging MedMNIST dataset, focusing particularly on the PathMNIST subset. This phase formed the core of our project, in which we focused intensively on experimenting with different pre-trained models. Our investigations extended to testing a wide range of hyperparameters as well as implementing different strategies for data preprocessing and augmentation. The complexity and challenges of the PathMNIST dataset required the use of more advanced techniques and approaches, representing a significant advance over our initial experiments with MNIST.\@

\subsection{MNIST}\label{mnist}
The ``modified National Institute of Standards and Technology'' dataset comprises a collection of 70,000 handwritten digits carefully divided into a training set of 60,000 images and a test set of 10,000 images. Each digit is represented in a grayscale image of 28$\times$28 pixels, and offers a wide range of styles and shapes. This dataset is widely recognized for its simplicity and effectiveness in benchmarking classification algorithms, making it an ideal starting point for those new to deep learning. Some examples of the dataset can be seeen in Figure~\ref{fig:MNIST}

\begin{figure}
    \centering
    \includegraphics[width=0.3\textwidth]{figures/MNIST.png}
    \caption{Example greyscale images of the MNIST dataset - CHANGE IMAGE - RENDER OWN!!!}\label{fig:MNIST}
\end{figure}

Chosen for our initial challenge, MNIST provided a fundamental platform to explore neural network basics and experiment with simple model architectures. It allowed us to grasp the essentials of model training, and hyperparameter testing.


\subsection{MedMNIST}\label{medmnist}

MedMNIST, a more specialized and challenging dataset than MNIST, is tailored for medical image classification tasks. It extends the concept of handwritten digit classification to a diverse range of medical imaging modalities, including dermatology or radiology. Unlike MNIST's uniform format, MedMNIST encompasses 12 subsets for 2D and 6 subsets for 3D data. For our project, we focused on the PathMNIST subset, which includes ``100.000 non-overlapping image patches from hematoxylin and eosin stained histological images, and a test dataset [$\ldots$] of 7.180 image patches from a different clinical center'' \citep{medmnistv1}. 

PathMNIST is designed to introduce the complexities of medical image analysis, featuring high-resolution images that require more sophisticated models and techniques. This subset challenges learners to apply and adapt advanced deep learning concepts, such as transfer learning and complex architectures like ResNet and Xception, to accurately classify medical images.

Engaging with MedMNIST, particularly the PathMNIST subset, allowed us to explore deep learning applications in a context with significant real-world implications. It provided an opportunity to extend our knowledge and skills beyond basic image classification, preparing us for the intricacies of analyzing medical images—a field where deep learning can have a profound impact on diagnostic processes and patient care.



Originally, the images were of high resolution (3 × 224 × 224 pixels), but for the purposes of this project, they were resized to 3 × 28 × 28 pixels to maintain consistency with the MNIST format and to make the dataset more manageable for training deep learning models. The NCT-CRC-HE-100K dataset was divided into training and validation sets in a 9:1 ratio, while the CRC-VAL-HE-7K dataset was used as the test set.

The PathMNIST subset of MedMNIST2D provides a unique challenge by introducing the complexity of medical image analysis. It requires the use of advanced deep learning techniques and models to accurately classify different types of tissue, making it an excellent progression from the simpler MNIST dataset. Working with PathMNIST allowed us to delve into the application of deep learning in medical diagnostics, showcasing the potential of these technologies to impact real-world healthcare outcomes significantly.

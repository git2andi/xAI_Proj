\section{Introduction}\label{intro}


The emergence of deep learning has led to unprecedented advances in various fields, including medical image analysis. This project seeks to explore the fundamental principles of deep learning and to leverage its potential in a practical setting. Our investigation is divided into two parts: First, we focus on the classification of handwritten digits using the MNIST dataset \citep{deng2012mnist}, followed by a more complicated challenge of classifying nine different tissue patterns within the PathMNIST dataset \citep{kather2018, kather2019}, a subset of MedMNIST \citep{medmnistv1}. These tasks not only serve as a basis for understanding the mechanisms of deep learning, but also highlight the impact of the application of neural networks in medical diagnostics, emphasizing the relevance of our research.

The initial phase of our project was dedicated to learning the basics using the MNIST dataset, which was chosen for its many resources and tutorials to help us get started with deep learning. In this phase, we developed a basic convolutional neural network (SimpleCNN) to introduce us to the architectures of neural networks and their ability to classify different digits.

The transition to the MedMNIST dataset, in particular the PathMNIST subset, represented a significant increase in the complexity of our project. This phase was crucial as it enabled us to apply and refine advanced techniques, such as experimenting with pre-trained models, testing a wide range of hyperparameters, and exploring different strategies for data pre-processing and augmentation. The PathMNIST subset was chosen to emphasize the critical importance of neural networks in medical image analysis. By tackling the classification of tissue patterns, we have not only explored some technical intricacies of deep learning, but also contributed to an area where such technologies could potentially revolutionize diagnostic methods.

This report is structured by first presenting the datasets used~-~MNIST and MedMNIST~-~which form the basis for a deeper investigation. We then explain some theoretical foundations that are crucial for understanding the methods used, including dropout layers and ReLU.\@ Next, we explain the architectures of different deep learning models that were investigated in the project, from our initial SimpleCNN to more complex models such as AlexNet, ResNet and Xception. Subsequently, we present the results obtained with each model, addressing the specific challenges. The discussion section provides a critical evaluation of our results and leads to a reflective conclusion about the lessons learned and the potential impact of our research on medical image analysis.

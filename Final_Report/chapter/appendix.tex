\section{Appendix}
\subsection{SimpleCNN architecture}\label{codeSnippets}
Initial version of our SimpleCNN, including two convulutional layers:\@

\begin{minted}[mathescape, linenos, fontsize=\scriptsize]{python}
class SimpleCNN(nn.Module):
    def __init__(self, num_classes=10):
        super(SimpleCNN, self).__init__()
        self.conv1 = nn.Sequential(
            nn.Conv2d(
                in_channels = 1,
                out_channels = 32,
                kernel_size = 3,
                stride=1,
                padding="same"
            ),
            nn.LeakyReLU(),
            nn.MaxPool2d(kernel_size=2),
        )
        self.conv2 = nn.Sequential(
            nn.Conv2d(32,64,3,1,"same"),
            nn.LeakyReLU(),
            nn.MaxPool2d(kernel_size=2),
        )
        self.out = nn.Linear(64*7*7, num_classes)

     def forward(self, x):
        x = self.conv1(x)
        x = self.conv2(x)
        x = x.view(-1, 64*7*7)
        output = self.out()
        return torch.log_softmax(output, dim=1)
\end{minted}


Structure of the improved version of the SimpleCNN using three convolutional layers, Batch normalization and Dropout:\@

\begin{minted}[mathescape, linenos, fontsize=\scriptsize]{python}
class SimpleCNN(nn.Module):
    def __init__(self, num_classes=10):
        super(SimpleCNN, self).__init__()
        self.conv1 = nn.Sequential(
            nn.Conv2d(1, 32, kernel_size=3, stride=1, padding="same"),
            nn.BatchNorm2d(32),
            nn.ReLU(),
            nn.MaxPool2d(kernel_size=2),
            nn.Dropout(0.25)
        )
        self.conv2 = nn.Sequential(
            nn.Conv2d(32, 64, kernel_size=3, stride=1, padding="same"),
            nn.BatchNorm2d(64),
            nn.ReLU(),
            nn.MaxPool2d(2),
            nn.Dropout(0.25)
        )
        self.conv3 = nn.Sequential(
            nn.Conv2d(64, 128, kernel_size=3, stride=1, padding="same"),
            nn.BatchNorm2d(128),
            nn.ReLU(),
            nn.MaxPool2d(2),
            nn.Dropout(0.25)
        )
        self.fc1 = nn.Linear(128 * 3 * 3, 256)
        self.fc_bn = nn.BatchNorm1d(256)
        self.dropout_fc = nn.Dropout(0.5)
        self.fc2 = nn.Linear(256, num_classes)

    def forward(self, x):
        x = self.conv1(x)
        x = self.conv2(x)
        x = self.conv3(x)
        x = x.view(-1, 128 * 3 * 3)
        x = F.relu(self.fc_bn(self.fc1(x)))
        x = self.dropout_fc(x)
        x = self.fc2(x)
        return torch.log_softmax(x, dim=1)
\end{minted}
    

Structure of the improved version of the SimpleCNN for the PathMNIST dataset:\@

\begin{minted}[mathescape, linenos, fontsize=\scriptsize]{python}
class SimpleCNN(nn.Module):
    def __init__(self, num_classes=10):
        super(SimpleCNN, self).__init__()
        self.conv1 = nn.Sequential(
            nn.Conv2d(3, 32, kernel_size=3, stride=1, padding="same"),
            nn.BatchNorm2d(32),
            nn.ReLU(),
            nn.MaxPool2d(kernel_size=2),
            nn.Dropout(0.25)
        )
        self.conv2 = nn.Sequential(
            nn.Conv2d(32, 64, kernel_size=3, stride=1, padding="same"),
            nn.BatchNorm2d(64),
            nn.ReLU(),
            nn.MaxPool2d(2),
            nn.Dropout(0.25)
        )
        self.conv3 = nn.Sequential(
            nn.Conv2d(64, 128, kernel_size=3, stride=1, padding="same"),
            nn.BatchNorm2d(128),
            nn.ReLU(),
            nn.MaxPool2d(2),
            nn.Dropout(0.25)
        )
        self.fc1 = nn.Linear(128 * 3 * 3, 256)
        self.fc_bn = nn.BatchNorm1d(256)
        self.dropout_fc = nn.Dropout(0.5)
        self.fc2 = nn.Linear(256, num_classes)

    def forward(self, x):
        x = self.conv1(x)
        x = self.conv2(x)
        x = self.conv3(x)
        x = x.view(-1, 128 * 3 * 3)
        x = F.relu(self.fc_bn(self.fc1(x)))
        x = self.dropout_fc(x)
        x = self.fc2(x)
        return torch.log_softmax(x, dim=1)
\end{minted}